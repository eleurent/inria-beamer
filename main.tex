\documentclass[slideopt,A4,showboxes,svgnames]{beamer}


%% list of packages here

\renewcommand{\sfdefault}{lmss}
\sffamily

\setbeamersize{text margin left=1cm,text margin right=1cm}
\setbeamerfont{alerted text}{series=\bfseries}
\setbeamerfont{example text}{series=\bfseries}

\usepackage[absolute,showboxes,overlay]{textpos}

\TPshowboxesfalse
\textblockorigin{0mm}{0mm}

\hypersetup{
allcolors=rouge_inria,
}

\newcommand{\GRAND}{\fontsize{100}{100}\selectfont}
\newcommand{\Grand}{\fontsize{80}{80}\selectfont}
\newcommand{\grand}{\fontsize{60}{60}\selectfont}
%% TODO : diminuer taille du titre dans la partie haute (cadre gris) : OK, on peut encore diminuer si besoin --> c'est Ok pour moi
%% Grossir le titre général sur slide 1 + le centrer (1 ou 2 lignes) : J'ai grossi et déplacé le texte sur la droite et un peu en haut. Faut-il encore le bouger ? --> C'est bien ainsi :-)

%% Le sous-titre en dessous puis l'auteur en dernier et en moins gros : Fait > VU
%% Slide 2 : on peut bien grossir le chiffre (01), slide 4 idem (02) et slide 6 aussi : Je viens de mettre trois tailles différentes, du plus petit au plus gros, sur les trois chapitres, dis-moi celle qui convient le mieux, ou s'il faut ajuster entre deux.
%% --> je ne vois qu'une taille mais c'est la bonne :-)
%% supprimer le texte dans la partie rouge en haut à gauche sur toutes les slides : OK
%% Logo seul sans baseline sur toutes les occurrences :OK
%% Réduire en taille ce logo : partout (sur la page de titre et en pied de page)? Ou juste en pied de page? --> juste en pied de page, en alignant horizontalement à droite, à l'aplomb du fond blanc

%% Un point rouge simple sans ombré pour les puces : OK
\title[titrecourt]{Titre sur une ligne\\ ou deux}
\subtitle{sous-titre}
\date[date]{date}
\author[Auteur]{Auteur}

\usetheme{inria}
%\usetheme{inria2}
%\usetheme{inria3}

\begin{document}

%%%%%%%%%%%%%%%%%%%%%%%%%%%%%%%%%%%%%%%%%%%%%%%%%%%%%%%
%% Titre de la présentation avec format Inria
%%%%%%%%%%%%%%%%%%%%%%%%%%%%%%%%%%%%%%%%%%%%%%%%%%%%%%%


\begin{frame}
    \titlepage
\end{frame}


%%%%%%%%%%%%%%%%%%%%%%%%%%%%%%%%%%%%%%%%%%%%%%%%%%%%%%%


%%%Plan de la présentation

%Chapitre 1 
 \frame{\tocpage}
 
 \section{Titre du chapitre}
 \frame{\sectionpage}
 
                  


\begin{frame}{Deux exemples de blocs}
    \begin{block}{}
 \begin{itemize}
    \item{Item 1 }
    \begin{itemize}
        \item {Sous-item 1}
 \item {Sous-item 2}
        \begin{itemize}
        \item {Sous-sous-item 1}
\item {Sous-sous-item 2}
    \end{itemize}
    \end{itemize}
    \item {Item 2}
    \item {Item 3}
    \end{itemize}
  \end{block}
      \begin{block}{Avec titre}
 \begin{itemize}
    \item{Item 1 }
    \item {Item 2}
    \item {Item 3}
    \end{itemize}
  \end{block}
\end{frame}




%Chapitre 2
\section{Titre du chapitre}
 \frame{\sectionpage}


\begin{frame}{Deux autres exemples de blocs}
    \begin{alertblock}{}
 \begin{itemize}
    \item{Item 1 }
    \item {Item 2}
    \item {Item 3}
    \end{itemize}
  \end{alertblock}
      \begin{alertblock}{Avec titre}
 \begin{itemize}
    \item{Item 1 }
    \item {Item 2}
    \item {Item 3}
    \end{itemize}
  \end{alertblock}
\end{frame}

%Chapitre 3
\section{Titre du chapitre}
 \frame{\sectionpage}

\begin{frame}{Deux autres exemples de blocs}
    \begin{exampleblock}{}
 \begin{itemize}
    \item{Item 1 }
    \item {Item 2}
    \item {Item 3}
    \end{itemize}
  \end{exampleblock}
      \begin{exampleblock}{Avec titre}
 \begin{itemize}
    \item{Item 1 }
    \item {Item 2}
    \item {Item 3}
    \end{itemize}
  \end{exampleblock}
\end{frame}
\begin{frame}{Titre}
Du \textcolor{gris_fonce_inria}{texte}  {en}  \textcolor{rouge_inria}{couleur} 
\end{frame}
\section{Conclusion}
\begin{frame}{Conclusion}
 
\end{frame}
\end{document}
